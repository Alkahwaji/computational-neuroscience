\documentclass[12pt]{article}
\usepackage{setspace}
\usepackage{pxfonts}
\usepackage{graphicx}
\usepackage{geometry}

\geometry{letterpaper,left=.5in,right=.5in,top=1in,bottom=.75in,headsep=5pt,footskip=20pt}

\title{Lecture 3 -- Integrate-and-fire neuron model}
\author{Computational Neuroscience Summer Program}
\date{June, 2011}

\begin{document}
\maketitle

\paragraph{Motivation.}  This lecture builds on the simple model
neuron that we developed in the last lacture by adding in action
potentials (APs).  Rather than modeling the biophysical basis of the
AP, in this model we manually cause the neuron to spike when its
membrane voltage reaches a threshold value.

\paragraph{Recap.}  Our working model is that neurons are like
capacitors connected to resistors -- the cell membrane stores charge,
which can leak out through ion channels.  As developed in the last
lecture, the equation we'll be using for all model neurons is:
\[
\tau_m\frac{dV}{dt} = E - V + R_mI_e
\]
Also from the previous lecture, we can solve for $V(t)$ as follows:
\[
V(t) = V_\infty + (V(0) - V_\infty) e^\frac{-t}{\tau_m}
\]

\paragraph{Running a simulation -- the ``integrate'' part of the
  model.}  Running a simulation entails
computing changes in the cell's membrane voltage for each iteration of the
simulation ($dt$ ms).  We can use the same equation as above, but
replace $V(t)$ with $V(t+dt)$ (i.e. the voltage in the next time step of
the simulation) and $V(0)$ (the starting voltage) with $V(t)$:
\[
V(t+dt) = \mathrm{(where~we~are~going)} +
\mathrm{(distance)}e^\frac{-t}{\tau_m} = V_\infty + (V(t) -
V_\infty)e^\frac{-t}{\tau_m}
\]
In practice, these computations work best for small values of $dt$ (in most cases
we'll use $dt \leq 0.1$ ms).

\paragraph{Running a simulation -- the ``fire'' part of the model.}
Now the integrate-and-fire model is almost entirely in place.  The
only thing we need to add is the rule that when $V(t+dt) \geq V_{thresh}$, simulate an action potential by 
setting $V(t) = V_{peak}$ and $V(t+dt) = V_{reset}$.  $V_{thresh}$ is
generally somewhere around -55 mV, $V_{peak}$ is around 40 mV, and
$V_{reset}$ is around -80 mV.  In the next lectures we'll discuss the biophysical
basis of why the neuron depolarizes (increases its membrane voltage)
suddenly during the start of the action potential and why the membrane
voltage becomes hyperpolarized (decreased) after the action potential is fired.


\paragraph{Computing firing rate.}  This is straightforward.  We can
simply count up the number of spikes that were fired during the
simulation (i.e. times when $V \geq V_{thresh}$ and divide by the
length of time we were simulating.

\paragraph{Analytic solution for firing rate.}  While the full integrate-and-fire
simulation is often useful (and is necessary if you want to model
things like spike timing), it turns out that there is an analytic
method for computing the expected firing rate of the model, given a
constant external current $I_e$.  We start with the equation for
finding the membrane voltage at time $t$:

\[
V(t) = V_\infty + (V(0) - V_\infty) e^\frac{-t}{\tau_m}
\]
We can then solve for $t$ as follows:
\[
V(t) - V_\infty = (V(0) - V_\infty) e^\frac{-t}{\tau_m}
\]

\[
\frac{V(t) - V_\infty}{(V(0) - V_\infty)} = e^\frac{-t}{\tau_m}
\]

\[
ln(\frac{V(t) - V_\infty}{V(0) - V_\infty}) = \frac{-t}{\tau_m}
\]

\[
\tau_mln(\frac{V(t) - V_\infty}{V(0) - V_\infty}) = -t
\]

\[
t = -\tau_mln(\frac{V(t) - V_\infty}{V(0) - V_\infty})
\]
Now let's suppose our model neuron has just fired a spike in the
previous timestep of our simulation.  We start by setting $V(0) =
V_{reset}$.  We next need to know how long it is until the neuron next
fires a spike (the inter-spike interval, $t_{isi}$ -- or, in other
words, the time $t$ at which $V(t) = V_{thresh}$ after starting at
$V(0) = V_{reset}$.  Plugging in the appropriate values, we can
compute $t_{isi}$ as follows:

\[
t_{isi} = -\tau_mln(\frac{V_{thresh} - V_\infty}{V_{reset} - V_\infty})
\]
Recall that $V_\infty = E + R_mI_e$.  Thus
\[
t_{isi} = -\tau_mln(\frac{V_{thresh} - (E + R_mI_e)}{V_{reset} - (E +
  R_mI_e)}) = -\tau_mln(\frac{V_{thresh} - E - R_mI_e}{V_{reset} - E - R_mI_e})
\]
The firing rate ($r_{isi}$) is the inverse of the inter-spike interval:
\[
r_{isi} = (-\tau_mln(\frac{V_{thresh} - E - R_mI_e}{V_{reset} - E - R_mI_e}))^{-1}
\]

\paragraph{Putting it all together.}  To run the simulation, start
with the basic model neuron equation:
\[
\tau_m\frac{dV}{dt} = E - V + R_mI_e
\]

Now solve for $dV$:
\[
\frac{dV}{dt} = \frac{E - V + R_mI_e}{\tau_m}
\]
\[
dV = (\frac{E - V + R_mI_e}{\tau_m})dt
\]
Start the simulation by setting $V(0) = E$.  With each timestep set $V(t+dt) = V(t) + dV$.
You'll need to re-compute $dV$ for each time-step given $V(t)$ and
$I_e(t)$ for the appropriate time $t$.  Remember to include the rule for
firing a spike (and resetting) when $V > V_{thresh}$ -- otherwise the
neuron won't fire spikes.



\paragraph{General MATLAB stuff.}
Set up your environment:
\begin{verbatim}
E = -70;        %mV
c_m = 10;       %nF / mm^2
r_m = 1;        %M ohm * mm^2
A = 0.025;      %mm^2
V_reset = -80;  %mV
V_thresh = -55; %mV
V_peak = 40;    %mV

dt = 0.1;       %ms
t = 1:dt:1000;  %ms
\end{verbatim}
Now loop:
\begin{verbatim}
V(1) = E;
for i = 2:length(t)
  if V(i-1) > V_thresh
    {fire a spike}
  else
    {compute dV}
    V(i) = V(i-1) + dV;
end
\end{verbatim}
For the problem set, you should write a function that runs the
integrate-and-fire model for a given set of parameters.  Your function
should at return the firing rate (computed numerically, not using the $r_{isi}$ equation).
You also might want to have it return the vector of $V$ over time,
depending on how you set up your code.

\end{document}


