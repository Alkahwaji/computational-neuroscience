\documentclass[11pt]{article}
\usepackage{setspace}
\usepackage{pxfonts}
\usepackage{graphicx}
\usepackage{geometry}

\geometry{letterpaper,left=.5in,right=.5in,top=0in,bottom=.75in,headsep=5pt,footskip=20pt}

\title{Problem Set 3 -- Integrate-and-fire neuron model}
\author{Computational Neuroscience Summer Program}
\date{June, 2011}

\begin{document}
\maketitle

In this problem set you will be building a simple integrate-and-fire
neuron.  You should assume a specific membrane capacitance of
$c_m = 10$ nF/mm\textsuperscript{2}, a specific membrane resistance of
$r_m = 1$ M$\Omega\cdot$mm\textsuperscript{2}, a resting membrane
potential of $E = -70$ mV, a reset potential of $V_{reset} = -80$ mV,
an action potential threshold of $V_{threshold} = -55$ mV, and a cell
surface area of $A = 0.025$ mm\textsuperscript{2}.  Write up your results in
a text editor of your choosing.  Include any relevant figures, your
Matlab code, and any other calculations related to the problem set.  You may work individually or in groups, but each student should hand in their own report.

\subsection*{Equations}

\begin{center}
\begin{tabular}{ll}
$C_m = A \cdot c_m$ & $R_m = \frac{r_m}{A}$\tabularnewline
$\tau_m = c_m \cdot r_m$ & $V_\infty = E + R_mI_{ext}$\tabularnewline
$V(t) = V_\infty + (V(0) - V_\infty)e^\frac{-t}{\tau_m}$ & $r_{isi} =
(\tau_m \mathrm{ln}(\frac{R_mI_{ext} + E - V_{reset}}{R_mI_{ext} + E -
  V_{threshold}}))^{-1}$\tabularnewline
$\tau_m\frac{dV}{dt} = E - V(t-1) + R_mI_{ext}$ & \tabularnewline
\end{tabular}
\end{center}

\subsection*{Problems}

\paragraph{1.} Model an integrate-and-fire neuron using the equations
above and the following rule: when the neuron's membrane voltage exceeds
$V_{threshold}$, set the voltage in that timestep to $V_{peak} = 40$ mV, and in the
next timestep set the voltage to $V_{reset}$.  Set $dt = 0.1$ ms.  Apply a square pulse of
0.5 nA from $t = 250$ ms until $t = 750$ ms in your simulation.  Use
Matlab's subplot command to plot the membrane voltage over time in the
top panel and $I_{ext}$ in the bottom panel (use the same time scale
for the horizontal axis of both plots).  No text is required for this
question; just include a plot.

\paragraph{2.} Compute the average firing rate (spikes per second) of the integrate-and-fire neuron
for the pulse interval you used in quesion 1 (500 ms).  Now plot simulated firing rate vs $r_{isi}$ for several values of $I_{ext}$ (use $I_{ext}$ between 0 and 1 nA).  How does the firing rate of the modeled neuron compare to the estimated firing
rate given by $r_{isi}$?  Note: the equation for $r_{isi}$ only holds if $V_\infty > V_{thresh}$; otherwise, $r_{isi} = 0$.

\paragraph{3.}  Starting from 0 nA, gradually increase the amount of external current
injected into the integrate-and-fire neuron in steps of 0.01 nA.  Keep the pulse duration constant at 500 ms.  What is the smallest amount of current you
can inject which will still result in an action potential?  Is there a
maximum firing rate this neuron can achieve?  Why or why not?

\paragraph{4.  Challenge problem.} Compute firing rate as a function of pulse duration, $I_{duration}$, using 20 durations between 10 and 500 ms.  Repeat this for several different values of $I_{ext}$ (try using 10 log-spaced values between 0.1 and 5 nA).  Explain what you see.  In particular, are the firing rate curves smooth or jagged?  Why?

\paragraph{5.} Vary the resting potential, specific
capacitance, specific resistance, and surface area variables.  How do
increases or decreases in these values affect firing rate of the integrate-and-fire neuron?  Explain
(try to stay at or under 1-2 sentences per variable).  Include plots for each of these variables.


\end{document}

