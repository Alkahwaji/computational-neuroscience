\documentclass[11pt]{article}
\usepackage{setspace}
\usepackage{pxfonts}
\usepackage{graphicx}
\usepackage{geometry}

\geometry{letterpaper,left=.5in,right=.5in,top=.5in,bottom=.75in,headsep=5pt,footskip=20pt}

\title{Problem Set 4 -- Integrate-and-fire neuron, Part 2}
\author{Computational Neuroscience Summer Program}
\date{June, 2009}

\begin{document}
\maketitle

\noindent Here you will add additional features to the integrate-and-fire neuron
model to make it more realistic.  Start with your basic neuron model from the
previous assignment using the same parameters and equations (i.e., specific
membrane capacitance of $c_m = 10$ nF/mm\textsuperscript{2}, a specific membrane
resistance of $r_m = 1$ M$\Omega\cdot$mm\textsuperscript{2}, a resting membrane
potential of $E = -70$ mV, a reset potential of $V_{reset} = -80$ mV, an action
potential threshold of $V_{threshold} = -55$ mV, and a cell surface area of
0.025 mm\textsuperscript{2}).

\subsection*{Equations}

\begin{center}
\begin{tabular}{ll}
  $C_m = A \cdot c_m$ & $R_m = \frac{r_m}{A}$\tabularnewline
  $\tau_m = c_m \cdot r_m$ & $V_\infty = E + R_mI_{ext}$\tabularnewline
  $V(t) = V_\infty + (V(0) - V_\infty)e^\frac{-t}{\tau_m}$ & $r_{isi} =
  (\tau_m \mathrm{ln}(\frac{R_mI_{ext} + E - V_{reset}}{R_mI_{ext} + E -
    V_{threshold}}))^{-1}$\tabularnewline
  $\tau_m\frac{dV}{dt} = E - V(t-1) + R_mI_{ext}$ & \tabularnewline
\end{tabular}
\end{center}

\subsection*{Problems}

\paragraph{1.  Spike-rate adaptation.}  When a standard neuron is injected with
current, the rate at which it emits spikes decreases over time---a phenomenon
called spike-rate adaptation.  Here we augment the basic integrate-and-fire
model with a potassium (K) channel so that it exhibits this phemenon.  This
equation describes the basic model:
\[ \tau_m\frac{dV}{dt} = E - V +R_mI_{ext} -G_{sra}(V-E_K),\] which has a new
additional term $-G_{sra}(V-E_K)$.  Here $G_{sra}$ indicates the
spike-rate-adaptation conductance and $E_K$ indicates the reversal potential,
$E_K=-70$.  In each iteration of the model, the spike-rate-adaptation
conductance relaxes to zero exponentially with time constant $\tau_{rsa}$:
\[\tau_{rsa}\frac{dG_{sra}}{dt}=-G_{sra}.\]  Whenever the neuron spikes, the
conductance $G_{sra}$ increases by $\Delta G_{sra}$:
\[G_{sra} \rightarrow G_{sra}+\Delta G_{sra}.\] When implementing this model,
use values: $\Delta G_{sra}=0.03$ and $\tau_{sra}=200$~ms.

Edit your integrate-and-fire model so that it implements this new spike-rate
adaptation feature.  Simulate one second of this model where there is a square
pulse at 0.4~nA from 250--750~ms.  Plot the output voltage $V$ and $G_{sra}$
as a function of time (use \texttt{subplot} to put each one on a separate plot).
Describe how the output of this model differs from the previous model that did
not have spike-rate adaptation.

\paragraph{2.  An absolute refractory period.}  Edit your standard integrate-and-fire model
so that it has a refractory period that prevents it from firing two spikes
within 5~ms of each other.  (To do this, have your model remember the timepoint
of the previous spike.)  For a 3-nA current injection at 250--750~ms, compare
this model's firing rate to a model that does not have a refractory period.  You
should show a plot of each model's outputs (change the axis limits to make
individual action potentials clear) and also compare the model's overall firing
rates.

Plot the mean firing of this model to 500-ms current pulses varying from in
amplitude 0 to 20 nA.  Describe this curve's shape.  Can you identify the
relation between this curve's shape and the model's parameters?

\paragraph{3.  A relative refractory period using the SRA equation.}
Rather than remembering the time of the last spike, can you use the spike-rate
adaptation equation to model a neuron that has a refractory period of
approximately 5 ms?  Hint: you'll use the same equation as in Question 1 but use
different values of the constants $\Delta G_{sra}$ and $\tau_{sra}$.

Test your model by plotting the membrane voltage of a cell in response to a
20-ms current pulse at 10 nA.  (A standard model without a refractory period
should emit approximately 20--25 spikes in response to this pulse).  Plot the output
of your model with a refractory period in response to this pulse, and report the
number of spikes that were emitted.  What parameters parameters did you use to
achieve this result?


\paragraph{4.  Challenge problem: oscillatory input.}  Use your spike-rate
adaptation with these parameters: $\Delta G_{sra}=3$ and $\tau_{sra}=50$~ms.
Rather than a square wave, stimulate this cell with an input current that
oscillates (i.e., a sine wave) between 0 and .5~nA at 10~Hz.  Using
\texttt{subplot}, plot the input current and the model's output on the same
timescale.

Next, simulate this model for various oscillating inputs ranging in frequencies
from 1~Hz to 50~Hz.  Plot the model's mean firing rate as a function
of firing rate.  Describe how the shape of this curve is controlled by the
model.

\end{document}

